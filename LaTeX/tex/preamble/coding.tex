% ~~~~~~~~~~~~~~~~~~~~~~~~~~~~~~~~~~~~~~~~~~~~~~~~~~~~~~~~~~~~~~~~~~~~~~~~
% verbatim packages
% ~~~~~~~~~~~~~~~~~~~~~~~~~~~~~~~~~~~~~~~~~~~~~~~~~~~~~~~~~~~~~~~~~~~~~~~~
 
%%% Doc: ftp://tug.ctan.org/pub/tex-archive/macros/latex/contrib/upquote/upquote.sty
\usepackage{upquote} % Setzt "richtige" Quotes in verbatim-Umgebung
 
%%% Doc: No Documentation;
  \usepackage{verbatim} %Reimplemntation of the original verbatim
 
%%% Doc: http://www.cs.brown.edu/system/software/latex/doc/fancyvrb.pdf
% \usepackage{fancyvrb} % Superior Verbatim Class

\IfPackageLoaded{fancyvrb}{
  \DefineShortVerb{\|} % Nur mit fancyvrb zusammen laden!
}
 
%% Listings Paket ------------------------------------------------------
%%% Doc: ftp://tug.ctan.org/pub/tex-archive/macros/latex/contrib/listings/listings-1.3.pdf
% f�r source code listings; verwenden mit \lstinline|...|, \begin{lstlisting}...\end{lstlisting} oder \lstinputlisting{file}
\usepackage{listings}
  \lstset{%
    language=R,
		basicstyle=\scriptsize\ttfamily,
    tabsize=3,
    showtabs=false,
    showspaces=false,
    showstringspaces=false,
    tab=\rightarrowfill,
    keywordstyle=\color{ListingsKeywordColor},
    identifierstyle=\color{ListingsIdentifierColor},
    commentstyle=\color{ListingsCommentColor},
    stringstyle=\color{ListingsStringColor},
    emphstyle=\color{ListingsEmphColor}\bfseries\underbar,
    frame=none,
    rulesepcolor=\color{ListingsRuleSepColor},
    numbers=left, 
    numberstyle=\tiny, 
    numbersep=5pt,
    captionpos=top,
    frame=single, %tb,
    firstnumber=1,
    stepnumber=5,
    numberfirstline=false,
    breaklines=true,
    breakatwhitespace=true,
    prebreak=\mbox{\,$\color{ListingsBreakSymbolColor}\mathbf{\hookleftarrow}$},
    mathescape=false,
    morekeywords={},
}
%  \lstset{%
%         basicstyle=\small\ttfamily, % Standardschrift
%         numbers=left,               % Ort der Zeilennummern
%         numberstyle=\tiny,          % Stil der Zeilennummern
%         stepnumber=2,               % Abstand zwischen den Zeilennummern
%         numbersep=5pt,              % Abstand der Nummern zum Text
%         tabsize=2,                  % Groesse von Tabs
%         extendedchars=true,         %
%         breaklines=true,            % Zeilen werden Umgebrochen
% %        keywordstyle=[1]\textbf,    % Stil der Keywords
% %        keywordstyle=[2]\textbf,    %
% %        keywordstyle=[3]\textbf,    %
% %        keywordstyle=[4]\textbf,    %
%         stringstyle=\color{stringcolor}, % Farbe der String
%         showspaces=false,           % Leerzeichen anzeigen ?
%         showtabs=false,             % Tabs anzeigen ?
%         showstringspaces=false      % Leerzeichen in Strings anzeigen ?
%  }
  \lstloadlanguages{% Check Dokumentation for further languages ...
          % [Sharp]C
%         [Visual]Basic
%         %Pascal
%         %C
%         %C++
%         %XML
%         %HTML
					R
  }
 
%%% Doc: ftp://tug.ctan.org/pub/tex-archive/macros/latex/contrib/examplep/eurotex_2005_examplep.pdf
% LaTeX Code und Ergebnis nebeneinander darstellen
%\usepackage{examplep}
%\usepackage[boxed,ruled]{algorithm2e}
%\usepackage[chapter]{algorithm}

%\usepackage{algorithmic}        % f�r Pseudo-Code-Algorithmen
%  \renewcommand{\algorithmicrequire}{\reallynopagebreak\textbf{Input:}}
%  \renewcommand{\algorithmicensure}{\reallynopagebreak\textbf{Output:}}
  
\usepackage[ruled]{algorithm2e}    %,algochapter      % f�r Algorithmus-Einbettungs-Umgebung

  \makeatletter
	%\newcommand\AND{\FuncSty{ AND }}
	\SetKw{AND}{and}
	\SetKw{not}{not}
	\DontPrintSemicolon%
\newenvironment{inlinealgorithm}[1]{%
  \medskip%
  \noindent\parbox{\linewidth}{\def\@fs@cfont{\bfseries}%
  \let\@fs@capt\relax%
  \par\noindent%
  \rule{\linewidth}{.8pt}%
  \vspace{-10pt}%
  \noindent\captionof{algorithm}{#1}%
  \vspace{-0.7\baselineskip}%
  \noindent\rule{\linewidth}{.4pt}%
  \vspace{0.2\baselineskip}}%
%  \vspace{-1.3\baselineskip}\reallynopagebreak%
%% My hack, as algorithmic uses the overall number of lines in the whole document for labels, instead of using the description from within a single algorithm
  \renewcommand*\theALC@unique{\theALC@line}%
  \reallynopagebreak%
}{%
  \vspace{-.75\baselineskip}%
  \rule{\linewidth}{.4pt}%
  \medskip%
	
}


\newenvironment{inlinelisting}[1]{%
  \def\@fs@cfont{\bfseries}%
  \let\@fs@capt\relax%
  \par\noindent%
  %\medskip%
  \rule{\linewidth}{.8pt}%
  \vspace{-10pt}%
%  \noindent\captionof{algorithm}{#1}%
  \vspace{-0.35\baselineskip}%
}{}
\makeatother


\usepackage{lipsum}